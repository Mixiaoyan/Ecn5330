\voffset=-.5in
\hoffset=-0.4in
\documentclass[11pt]{article}
\renewcommand{\textwidth}{6.0 in}
%\renewcommand{\textheight}{9.75 in}

\usepackage{graphicx}
\usepackage{lscape}
\usepackage{amsmath}
\usepackage{mathrsfs}
\usepackage[pdftex]{color}

\usepackage{textcomp}

\textheight 8.6 in
\flushbottom

\setlength{\parindent}{0in}

\begin{document}\pagestyle{empty}

\textbf{Elementary Probability Review Continued} \\
\textbf{Economics 5330, Fall 2016} \\
\vspace{3.5mm}


This is a review of elementary probability that will be useful for our study of financial econometrics. It is based on coverage Wooldridge (2004).

\vspace{3.5mm}

The \textbf{cumulative distribution function} (\textbf{CDF}) of the random variable $X$ is:

\begin{equation*}
F(X) = P(X \leq x)
\end{equation*}

\vspace{2mm}

For discrete random variables it is obtained by summing the PDF over all values $x_{j}$ such that $x_{j} \leq x$.

\vspace{2mm}

For a continuous random variable, $F(X)$ is the area under the PDF, $f(x)$ to the left of $x$.

\vspace{2mm}

Because it is a probability, $0 \leq F(X) \leq 1$.

\vspace{2mm}

If $x_{1} < x_{2}$ then $P(X \leq x_{1}) \leq P(X \leq x_{2})$, that is $F(x_{1}) \leq F(x_{2})$.

\vspace{2mm}

Two important properties of CDFs that are useful for computing probabilities are the following:

\begin{itemize}
 \item For and number $c$, $P(X > c) = 1 - F(c)$
 \item For any numbers $a$ and $b$, $P(a \leq X \leq b) = F(b) - F(a)$
\end{itemize}

\vspace{2mm}

For continuous random variables the inequalities in probability statements are not strict:

\begin{equation*}
P(X \geq c) = P( > c)
\end{equation*}


\begin{align*}
P(a < X < b) &= P(a \leq X \leq b) \\
             &= P(a \leq X < b)    \\
             &= P(a < X \leq b)
\end{align*}


\vspace{2mm}

Let $X$ and $Y$ be discrete random variables. Then for $(X,Y)$ a \textbf{joint distribution} which is
fully described by the \textbf{joint probability density function} of $(X,Y)$:

\begin{equation*}
f_{XY}(x, y) = P(X = x, Y = y)
\end{equation*} 

\vspace{2mm}

$X$ and $Y$ are said to be independent if, and only if:

\begin{equation*}
f_{XY}(x,y) = f_{X}(x) f_{Y}(y) \quad \mbox{for every $x$ and $y$}
\end{equation*}

where $f_{X}$ is the PDF of the random variable $X$, and $f_{Y}$ is the PDF of random variable $Y$.

\vspace{2mm}

$f_{X}$ and $f_{Y}$ are referred to as the \textbf{marginal probability density functions}.

\vspace{2mm}

The discrete case is the easiest to grok. If $X$ and $Y$ are discrete and independent then

\begin{equation*}
P(X=x, Y=y) = P(X=x)P(Y=y)
\end{equation*}

\vspace{2mm}

Note: If $X$ and $Y$ are independent then finding the joint PDF only requires knowledge of
$P(X=x)$ and $P(Y=y)$

\vspace{2mm}

Example: Consider a basketball player shooting two free throws. Let $X$ be the Bernoulli random variable
equal to $1$ if he makes the first free throw, and $0$ otherwise. Let $Y$ be the Bernoulli random
variable equal to $1$ if he makes the second free throw. Suppose that he is an $80\%$ free throw 
shooter, so that $P(X=1) = P(Y=1) = 0.80$. What is the probability of making both free throws?

\vspace{2mm}

If $X$ and $Y$ are independent: $P(X=1, Y=1) = P(X=1)P(Y=1) = (0.8) ( 0.8) = 0.64$. Thus, a $64\%$
chance of making both.

\vspace{2mm}

Independence is often reasonable in more complicated situations. In the airline example, suppose that 
$n$ is the number of reservations booked. For each $i = 1, 2, \ldots, n$ let $Y_{i}$ denote the Bernoulli
random variable indicating whether or not customer $i$ shows up for the flight. 

\vspace{2mm}

Let $\theta$ again denote the probability of success (showing up for the reservation). 
Each $Y_{i} \sim \mbox{Bernoulli($\theta$)}$.

\vspace{2mm}

The variable of primary interest is the total number of customers showing up out of the $n$ reservations: 
call this $X$.

\begin{equation*}
X = Y_{1} + Y_{2} + \ldots + Y_{n}
\end{equation*}

\vspace{2mm} 

Assume that $P(Y_{i} = 1) = \theta$ for every $Y_{i}$, and further that they $Y_{i}$ are independent.
Then $X$ has a \textbf{binomial distribution}, which we write in shorthand as: $X \sim \mbox{Binomial($n$, $\theta$)}$.
The binomial PDF is the following:

\begin{equation*}
f(x) = {n \choose x} \theta^{x} (1-\theta)^{n-x} \quad \mbox{for $x = 0, 1, 2, \ldots, n$}
\end{equation*}

\vspace{2mm}

Note: ${n \choose x} = \frac{n!}{x! (n-x)!}$, and is read as ``n choose x''.

\vspace{2mm}

Example: If the flight has $100$ seats and $n = 120$ and $\theta = 0.85$ then:

\begin{equation*}
P(X > 100) = P(X=101) + P(X=102) + \ldots + P(X=120)
\end{equation*}

\vspace{2mm}

In econometrics we are usually interested in how one variable $Y$ is related to one or more other variables. For now,
consider only one such variable $X$. What we can know about how $X$ affects $Y$ is contained in the \textbf{conditional
distribution} of $Y$ given $X$. This information is summarized in the
\textbf{conditional probability distribution function}:

\begin{equation*}
f_{Y|X}(y|x) = \frac{f_{XY}(x, y)}{f_{X}(x)}
\end{equation*}

\vspace{2mm}

In the discrete case: $f_{Y|X}(y|x) = P(Y=y|X=x)$, which we read as the probability that $Y=y$ given that $X=x$.

\vspace{2mm}

If $X$ and $Y$ are independent, then the knowledge of $X$ tells us nothing about $Y$:

\begin{align*}
f_{Y|X}(y|x) &= f_{Y}(y) \quad \mbox{and} \\
f_{X|Y}(x|y) &= f_{X}(x)
\end{align*}


\vspace{2mm}

Example: Free throw shooting again. Assume the conditional PDF is given by the following:

\begin{itemize}
 \item[] $f_{Y|X}(1|1) = 0.85$, and $f_{Y|X}(0|1) = 0.15$.
 \item[] $f_{Y|X}(1|0) = 0.70$, and $f_{Y|X}(0|0) = 0.30$.
\end{itemize}

These are not independent. The probability of making the second free throw depends on whether or not the first
free throw was made. We can calculate $P(X=1, Y=1)$ if we know $P(X=1)$. Assume the probability of making the
first free throw is $P(X=1) = 0.80$. Then:

\begin{align*}
P(X=1, Y=1) &= P(Y=1|X=1) \times P(X=1) \\
            &= (0.85) \times (0.80)     \\
            &= 0.68
\end{align*}

\vspace{2mm}

The \textbf{expected value} is a measure of central tendency. It is one of the most important probabilistic 
concepts in econometrics. If $X$ is a random variable the \textbf{expected value} (or expectation) of $X$,
denoted $E(X)$ and sometimes $\mu$, is a weighted average of all possible values of $X$. The weights are 
determined by the PDF.

\vspace{2mm}

Consider the case of a discrete random variable. Let $f(x)$ denote the PDF of $X$. The expected value of $X$
is the weighted average:

\begin{equation*}
E(X) = x_{1}f(x_{1}) + x_{2}f(x_{2}) + \ldots + x_{k}f(x_{k}) = \sum\limits_{j=1}^{k} x_{j}f(x_{j})
\end{equation*}

\vspace{2mm}

Example: Suppose $X$ takes on the values $-1$, $0$, and $2$ with probabilities $\frac{1}{8}$, $\frac{1}{2}$,
$\frac{3}{8}$. Then 

\begin{equation*}
E(X) = (-1)(\frac{1}{8}) + (0)(\frac{1}{2}) + (2)(\frac{3}{8}) = \frac{5}{8}
\end{equation*}

\vspace{2mm}

Note: $E(X)$ can take on values that are not even possible outcomes of $X$.

\vspace{2mm}

If $X$ is a continuous random variable then 

\begin{equation*}
E(X) = \int\limits_{-\infty}^{\infty} xf(x)dx
\end{equation*}

This is still interpreted as a weighted average.

\vspace{2mm}

Given a random variable $X$ and a function $g(\cdot)$, we can create a new random variable $g(X)$. For example, 
if $X$ is a random variable, then so is $X^{2}$ or $log(X)$ (for $x > 0$).

\vspace{2mm}

The expected value of $g(X)$ is given by

\begin{equation*}
E[g(X)] = \sum\limits_{j=1}^{k} g(x_{j}) f_{X}(x_{j})
\end{equation*}

or

\begin{equation*}
E[g(X)] = \int\limits_{-\infty}^{\infty} g(x)f_{X}(x)dx
\end{equation*}

\vspace{2mm}

Example: For the random variable above let $g(X) = X^{2}$. Then 

\begin{equation*}
E(X^{2}) = (-1)^{2}(\frac{1}{8}) + (0)^{2}(\frac{1}{2}) + (2)^{2}(\frac{3}{8}) = \frac{13}{8}
\end{equation*}

Note: $E[g(X)] \neq g[E(X)]$.

\vspace{2mm}

Properties of Expected Values:

\begin{itemize}
 \item[] \textbf{Property E1:} For any constant $c$, $E(c) = c$.
 \item[] \textbf{Property E2:} For any constants $a$ and $b$, $E(aX + b) = aE(X) + b$.
 \item[] \textbf{Property E3:} If ${a_{1}, a_{2}, \ldots, a_{n}}$ are constants and ${X_{1}, X_{2}, \ldots, X_{n}}$ are
      random variables then:
 \begin{itemize}
  \item $E(a_{1}X_{1} + a_{2}X_{x} + \ldots + a_{n}X_{n}) = a_{1}E(X_{1}) + a_{2}E(X_{2}) + \ldots + a_{n}E(X_{n})$
  \item Or $E(\sum\limits_{i=1}^{n} a_{i}X_{i}) = \sum\limits_{i=1}^{n} a_{i}E(X_{i})$
  \item A special case is when each $a_{i} = 1$ so that
          $E(\sum\limits_{i=1}^{n} E(X_{i})) = \sum\limits_{i=1}^{n} E(X_{i})$, or in other words the expected value
          of a sum, is the sum of the expected values.
 \end{itemize}
\end{itemize}

\vspace{2mm}

Example: Expected revenue at a pizzeria. $X_{1}$, $X_{2}$, and $X_{3}$ are the number of small, medium, and large
pizzas sold during the day. Suppose $E(X_{1}) = 25$, $E(X_{2}) = 57$, and $E(X_{3}) = 40$. Prices are $\$5.50$ for
a small, $\$7.60$ for a medium, and $\$9.15$ for a large. Then expected revenue is the following 

\begin{align*}
E(5.50X_{1} + 7.60X_{2} + 9.15X_{3}) &= 5.50E(X_{1}) + 7.60E(X_{2}) + 9.15E(X_{3}) \\
                                     &= 5.50(25) + 7.60(57) + 9.15(40) \\
                                     &= 936.70 
\end{align*} 

The outcome on any given day will differ from this, but this is the expected revenue.

\vspace{2mm}

If $X \sim \mbox{Binomial($n$, $\theta$)}$ then $E(X) = n\theta$. The expected number of successes in $n$ 
Bernoulli trials is $n\theta$. We can see this by writing

\begin{equation*}
X = Y_{1} + Y_{2} + \ldots + Y_{n} \quad \mbox{where each $Y_{i} \sim$ Bernoulli($\theta$)}
\end{equation*}

Then

\begin{align*}
E(X) &= \sum\limits_{i=1}^{n} E(Y_{i}) \\
     &= \sum\limits_{i=1}^{n} \theta   \\
     &= n\theta
\end{align*}

\vspace{2mm}

Example: Consider the airline problem with $n=120$ and $\theta = 0.85$. Then 
$E(X) = n\theta = 120(0.85) = 102$, which is too many.

\vspace{2mm}

The \textbf{median} is another measure of central tendency. If $X$ is continuous then the median is
the value $m$ such that one--half of the area under the PDF is to the left of $m$, and one--half is to 
the right of $m$.

\vspace{2mm}

If $X$ is discrete and takes on an odd number of finite values, the median is obtained by ordering the
possible outcomes of $X$ and selecting the middle value.

\vspace{2mm}

Example: For the sample $\{-4, 0, 2, 8, 10, 13, 17\}$ the median is $8$.

\vspace{2mm}

If $X$ takes on an even number of values, then the median is the average of the two middle values.

\vspace{2mm}

Example: For the sample $\{-5, 3, 9, 17\}$ the median is $\frac{3+9}{2} = 6$.

\vspace{2mm}

For a random variable let $E(X) = \mu$. There are various ways to measure how far $X$ is from its expected value.
One of the simplest is the squared distance:

\begin{equation*}
(X - \mu)^{2}
\end{equation*} 

This eliminates the sign, which corresponds with our intuitive notion of a distance measure. It treats values 
above and below $\mu$ symmetrically.

\vspace{2mm}

The \textbf{variance} is defined as follows:

\begin{equation*}
Var(X) = E[(X - \mu)^{2}]
\end{equation*}

The variance is sometimes denoted by $\sigma_{X}^{2}$ or just $\sigma^{2}$ when the random variable is understood
to be $X$.

\vspace{2mm}

Note: 

\begin{align*}
\sigma^{2} &= E(X^{2} - 2X\mu + \mu^{2}) \\
            &= E(X^{2}) - 2\mu^{2} + \mu^{2} \\
            &= E(X^{2}) - \mu^{2}
\end{align*}

\vspace{2mm}

Example: If $X \sim$ Bernoulli($\theta$) we know that $E(X) = \theta$. Since $X^{2} = X$ it follows that
$E(X^{2}) = \theta$. Then $Var(X) = E(X^{2}) - \mu^{2} = \theta - \theta^{2} = \theta(1 - \theta)$.

\vspace{2mm}

Properties of variance:

\begin{itemize}
 \item[] \textbf{Property VAR1:} $Var(X) = 0$ if, and only if for every $c$ such that $P(X=c) = 1$, in which
         case $E(X) = c$.
 \item[] \textbf{Property VAR2:} For constants $a$ and $b$ $Var(aX + b) = a^{2} Var(X)$.
\end{itemize}

\vspace{2mm}

The \textbf{standard deviation} is related to the variance as follows: $sd(X) = \sqrt{Var(x)}$. The standard 
deviation is often denoted $\sigma_{x}$ or just $\sigma$.

\vspace{2mm}

Properties of the standard deviation:

\begin{itemize}
 \item[] \textbf{Property SD1:} For a constant $c$, $sd(c) = 0$.
 \item[] \textbf{Property SD2:} For constants $a$ and $b$ $sd(aX + b) = |a|sd(X)$.
\end{itemize}

\vspace{2mm}

Given a random variable $X$, we can define a new random variable $Z$ by 

\begin{equation*}
Z = \frac{X - \mu}{\sigma}
\end{equation*}

or $Z = aX + b$ where $a = \frac{1}{\sigma}$ and $b = \frac{-\mu}{\sigma}$. Then 
$E(Z) = aE(X) + b = \frac{\mu}{\sigma} - \frac{\mu}{\sigma} = 0$.

\vspace{2mm}

The variance is $Var(Z) = a^{2}Var(X) = \frac{\sigma^{2}}{\sigma^{2}} = 1$. Thus the new random variable
has $\mu = 0$ and $\sigma^{2} = 1$. This is known as \textbf{standardizing} a random variable.

\vspace{2mm}

Example: Suppose $E(X) = 2$ and $Var(X) = 9$ then $Z = \frac{X - 2}{3}$.

\vspace{2mm}

While the joint distribution completely describes the relationship between two random variables it is often useful
to have a summary measure of how, on average, two random variables vary with one another.

\vspace{2mm}

The \textbf{covariance} is defined as follows:

\begin{equation*}
Cov(X,Y) = E[(X-\mu_{X})(Y - \mu_{Y})]
\end{equation*}

The covariance is often denoted by $\sigma_{XY}$. If $\sigma{XY} > 0$ then on average when $X$ is above its mean $Y$
is also above its mean. If $\sigma_{XY} < 0$ then on average when $X$ is above its mean $Y$ is below its mean, and vice
versa.

\vspace{2mm}

Note: 

\begin{align*}
Cov(X,Y) &= E[(X - \mu_{X})(Y - \mu_{Y})] \\
         &= E[(X - \mu_{X}) Y]            \\
         &= E(XY) - \mu_{X}\mu_{Y}
\end{align*}


\vspace{2mm}

\newpage
Properties of covariance:

\begin{itemize}
 \item[] \textbf{Property COV1:} If $X$ and $Y$ are independent then $Cov(X,Y) = 0$. Note: the converse
         is not true. Zero $Cov(X, Y)$ does not imply independence.
 \item[] \textbf{Property COV2:} For any constants $a_{1}$, $b_{2}$, $a_{2}$, and $b_{2}$ 
         $Cov(a_{1}X + b_{1}, a_{2}Y + b_{2}) = a_{1}a_{2}Cov(X,Y)$.
 \item[] \textbf{Property COV3:} $|Cov(X,Y)| \leq sd(X)sd(Y)$.
\end{itemize}

\vspace{2mm}

Note: property COV2 suggests that  $Cov(X,Y)$ depends upon how the random variables are measured, not only
on how strongly they are related. In other words, scale matters for $Cov(X,Y)$.

\vspace{2mm}

The \textbf{correlation coefficient} is defined as 

\begin{equation*}
Corr(X,Y) = \frac{Cov(X,Y)}{sd(X)sd(Y} = \frac{\sigma_{XY}}{\sigma_{X}\sigma_{Y}}
\end{equation*}

The correlation coefficient is sometimes denoted by $\rho_{XY}$.

\vspace{2mm}

Properties of correlation:

\begin{itemize}
 \item[] \textbf{Property CORR1:} $-1 \leq Corr(X,Y) \leq 1$.
 \item[] \textbf{Property CORR2:} For constants $a_{1}$, $b_{1}$, $a_{2}$, and $b_{2}$ with $a_{1}a_{2} > 0$
         $Corr(a_{1}X + b_{1}, a_{2}Y + b_{2}) = Corr(X,Y)$. If $a_{1}a_{2} < 0$ then 
         $Corr(a_{1}X + b_{1}, a_{2}Y + b_{2}) = -Corr(X,Y)$.
\end{itemize}

\vspace{2mm}

With covariance and correlation defined we state further properties of the variance:

\begin{itemize}
 \item[] \textbf{Property VAR3:} For constants $a$ and $b$, $Var(aX + bY) = a^{2} Var(X) + b^{2} Var(Y) + 2abCov(X,Y)$.
 \item[] \textbf{Property VAR4:} If $\{X_{1}, X_{2}, \ldots, X_{n}\}$ are pairwise uncorrelated and 
         $\{a_{i}: i=1, \ldots, n \}$ are constants then
         $Var(\sum\limits_{i=1}^{n} a_{i} X_{i}) = \sum\limits_{i=1}^{n} a_{i}^{2} Var(X_{i})$.
\end{itemize}

\vspace{2mm}

The \textbf{conditional mean} is defined as follows:

\begin{equation*}
E(Y|x) = \sum\limits_{j=1}^{m} y_{j} f_{Y|X}(y_{j}|x)
\end{equation*}

Example: Let $(X,Y)$ represent the population of all working individuals, where $X$ is years of education and $Y$
is hourly wages. Then $E(Y|X=12)$ is the average hourly wage for all the people in the population with $12$ years
of education (roughly high school education). $E(Y|X=16)$ is the average hourly wage for all people with $16$ years
of education. 

\vspace{2mm}

A typical situation in econometrics will look like the following:

\begin{equation*}
E(WAGE|EDUC) = 1.05 + 0.45 EDUC
\end{equation*}

If this linear relationship holds then for $8$ years of education the expected hourly wage is $1.05 + 0.45(8) = 4.65$
of $\$4.65$ per hour.

\vspace{2mm}

Properties of conditional expectations:

\begin{itemize}
 \item[] \textbf{Property CE1:} $E[c(X)|X] = c(X)$ for any function $c(X)$. In other words, functions act
         as constants. For example, $E[X^{2}|X] = X^{2}$. If we know $X$ we also know $X^{2}$.
 \item[] \textbf{Property CE2:} For funtions $a(X)$ and $b(X)$, $E[a(X)Y + b(X)|X] = a(X)E(Y|X) + b(X)$.
         For example, consider the random variable $XY + 2X^{2}$. $E(XY + 2X^{2}|X) = XE(Y|X) + 2X^{2}$.
 \item[] \textbf{Property CE3:} If $X$ and $Y$ are independent then $E(Y|X) = E(Y)$.
 \item[] \textbf{Property CE4:} $E[E(Y|X)] = E(Y)$. This is known as the Law of Iterated Expectations.
 \item[] \textbf{Property CE5:} $E(Y|X) = E[E(Y|X,Z)|X]$.
 \item[] \textbf{Property CE6:} If $E(Y|X) = E(Y)$ then $Cov(X,Y) = 0$ and $Corr(X,Y) = 0$.
\end{itemize}

\vspace{2mm}

The \textbf{conditional variance} is defined as follows:

\begin{equation*}
Var(Y|X=x) = E(Y^{2}|X) - [E(Y|X)]^{2}
\end{equation*}

\vspace{2mm}

Properties of conditional variance:

\begin{itemize}
 \item[] \textbf{Property CV1:} If $X$ and $Y$ are independent then $Var(Y|X) = Var(Y)$.
\end{itemize}

\vspace{2mm}

The \textbf{normal probability density function} is defined as follows:

\begin{equation*}
f(x) = \frac{1}{\sigma \sqrt{2\pi}} \exp{\frac{-(X-\mu)^{2}}{2\sigma^{2}}}, \quad \mbox{for $-\infty < x < \infty$}
\end{equation*}

where $E(X) = \mu$ and $Var(X) = \sigma^{2}$. When is a random variable is normally distributed we write 
$X \sim N(\mu, \sigma^{2})$.

\vspace{2mm}

A special case is the \textbf{standard normal distribution}, which is defined as follows:

\begin{equation*}
\phi(z) = \frac{1}{\sqrt{2\pi}} \exp{\frac{-z^{2}}{2}}, \quad \mbox{for $-\infty < z < \infty$} 
\end{equation*}

The standard normal cumulative distribution function is denoted by $\Phi(z) = P(Z \leq z)$. Using some basic 
facts from probability we arrive at the following helpful formulas:

\begin{align*}
P(Z > z)  &= 1 - \Phi(z) \\
P(Z < -z) &= P(Z > z) \\
P(a \leq Z \leq b) &= \Phi(b) - \Phi(a)
\end{align*}

\vspace{2mm}
\newpage

Properties of the normal distribution:

\begin{itemize}
 \item[] \textbf{Property NORMAL1:} If $X \sim N(\mu, \sigma^{2})$ then $\frac{(X - \mu)}{\sigma} \sim N(0, 1)$.
 \item[] \textbf{Property NORMAL2:} If $X \sim N(\mu, \sigma^{2})$ then $aX + b \sim N(a\mu + b, a^{2}\sigma^{2})$.
 \item[] \textbf{Property NORMAL3:} If $X$ and $Y$ are jointly normally distributed, then they are independent if, 
         and only if $Cov(X,Y) = 0$.
 \item[] \textbf{Property NORMAL4:} Any linear combination of independent, identically distributed normal random
         variables has a normal distribution.
\end{itemize}

Example: Let $X_{i}$ for $i = 1, 2, \mbox{and}, 3$, be independent random variables distributed as $N(\mu, \sigma^{2})$.
Define $W = X_{1} + 2X_{2} - 3X_{3}$. Then $W$ is normally distributed. We can solve for the mean and variance as
follows: 

\begin{align*}
E(W) &= E(X_{1}) + 2E(X_{2}) - 3E(X_{3}) = \mu + 2\mu - 3\mu = 0 \\
Var(W) &= Var(X_{1}) + 4Var(X_{2}) + 9Var(X_{3}) = 16\sigma^{2}
\end{align*}

\vspace{2mm}

The \textbf{chi--square distribution} is obtained directly from independent, standard normal random variables.
Let $Z_{i}$, $i = 1, 2, \ldots, n$, be independent random variables, each distributed as standard normal. Define a
new random variable as the sum of the squares of the individual $Z_{i}$:

\begin{equation*}
X = \sum\limits_{i=1}^{n} Z_{i}^{2}
\end{equation*}

The new random variable $X$ has a \textbf{chi--square distribution} with $n$ \textbf{degrees of freedom}. This is 
often written as $X \sim \chi_{n}^{2}$.

\vspace{2mm}

The \textbf{$t$ distribution} is a workhorse in classical statistics and econometrics. A $t$ distribution is
obtained from a standard normal and a chi--square random variable. Let $Z$ have a standard normal distribution
and let $X$ have a chi-square distribution with $n$ degrees of freedom. Also assume that $Z$ and $X$ are independent.
Then the following random variable

\begin{equation*}
T = \frac{Z}{\sqrt{Z/n}}
\end{equation*} 

has a $t$ distribution with $n$ degrees of freedom. This is denoted by $T \sim t_{n}$. The $t$ distribution gets
its degrees of freedom from the chi--square random variable.

\vspace{2mm}

Another important distribution for statistics and econometrics is the \textbf{$F$ distribution}. To define an $F$
random variable, let $X_{1} \sim \chi_{k_{1}}^{2}$ and $X_{2} \sim \chi_{k_{2}}^{2}$ and assume that $X_{1}$ and
$X_{2}$ are independent. Then, the random variable

\begin{equation*}
F = \frac{X_{1}/k_{1}}{X_{2}/k_{2}}
\end{equation*}

has an $F$ distribution with $(k_{1}, k_{2})$ degrees of freedom. We denote this as $F \sim F_{k_{1}, k_{2}}$. The 
order of the degrees of freedom is important. $k_{1}$ is the \emph{numerator degrees of freedom} and $k_{2}$ is 
the \emph{denominator degrees of freedom}.

\end{document}

