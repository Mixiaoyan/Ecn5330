\voffset=-.5in
\hoffset=-0.4in
\documentclass[11pt]{article}
\renewcommand{\textwidth}{6.0 in}
%\renewcommand{\textheight}{9.75 in}

\usepackage{graphicx}
\usepackage{lscape}
\usepackage{amsmath}
\usepackage{mathrsfs}
\usepackage[pdftex]{color}

\usepackage{textcomp}

\textheight 8.6 in
\flushbottom

\setlength{\parindent}{0in}

\begin{document}\pagestyle{empty}

\textbf{Economics 5330, Fall 2016}

\vspace{3.5mm}

\textbf{Review Notes: Vector Autoregressive Models}
\vspace{3.5mm}

\textbf{\textit{N.B.: I produced these notes original for my advanced econometrics class, the presentation uses some
advanced concepts and is expressed in matric form. But for those would like to have a brief overview at a slightly
deeper level on Vector Autoregressive Models this set of notes might prove helfpul.}}


This is a review of Vector Autoregressive Models (VARs).  The purpose is to
provide a review and to fix notation and ideas. We will assume, unless 
stated otherwise, that we are dealing with a stationary VAR$(p)$ process. These
short notes are not meant to replace your classroom notes, but rather to supplement,
summarize, and to correct notation.  I hope they are helpful to you.

\vspace{10mm}
\textbf{Reduced-Form VARs}

\vspace{3.5mm}
A reduced-form VAR$(p)$ consists of $K$ endogenous variables, 
$y_{t} = (y_{1t}, y_{2t}, \ldots, y_{Kt})^{\prime}$, and can be written as:

\begin{equation*}
y_{t} = \nu + A_{1} y_{t-1} + A_{2} y_{t-2} + \cdots + A_{p} y_{t-p} + u_{t}
\end{equation*} 

\vspace{3.5mm}
For $K=2$ (bivariate) a VAR$(p)$ model in matrix form is the following:

\vspace{3mm}
\begin{equation*}
\left[ \begin{array}{c} y_{1t} \\ y_{2t} \end{array} \right] = \left[\begin{array}{c} \nu_{1} \\ \nu_{2} \end{array}\right]
+ \begin{bmatrix} a_{11}^{(1)} & a_{12}^{(1)} \\ a_{21}^{(1)} & a_{22}^{(1)} \end{bmatrix} \times 
\left[\begin{array}{c} y_{1t-1} \\ y_{2t-1} \end{array} \right] + 
\cdots + 
\begin{bmatrix} a_{11}^{(p)} & a_{12}^{(p)} \\ a_{21}^{(p)} & a_{22}^{(p)} \end{bmatrix} \times
\left[\begin{array}{c} y_{1t-p} \\ y_{2t-p} \end{array} \right] +  
\left[\begin{array}{c} u_{1t} \\ u_{2t} \end{array} \right]
\end{equation*}

\vspace{3mm}
We can obtain the variance-covariance matrix as $E[u_{t}u_{t}^{\prime}] = \Omega$.

\vspace{3mm}
\textbf{Note:} A VAR model is a Seemingly Unrelated Regressions (SUR) model with each equation containing the same regressors. 
In this special case we noted that there is no additional efficiency gained by systems estimation (using GLS).  So the optimal
estimation strategy is to apply OLS equation-by-equation to VAR system. Estimating the VAR system in this form is particularly 
convenient.

\vspace{10mm}
\textbf{Vector Moving Average Form of the Reduced-Form VAR} \\

\vspace{3.5mm}
We noted that every stationary VAR can be represented as an infinite-order Vector Moving Average (VMA) model.  The model in this
form is given by the following:

\vspace{3.5mm}
\begin{equation*}
y_{t} = \mu + \sum\limits_{i=0}^{\infty} \Phi_{i} u_{t-i}
\end{equation*}

in which $\mu = \sum\limits_{i=0}^{\infty} \Phi_{i} \nu$.

\vspace{3mm}
\textbf{Note:} Typically the model is estimated in the VAR$(p)$ form and ``inverted'' to the VMA form.

\newpage
To go from the VAR$(p)$ form to the VMA form one applies the following recursions:\

\begin{equation*}
\Phi_{s} = \sum\limits_{j=1}^{s} \Phi_{s-j} A_{j}, \mbox{\quad for $s = 1, 2, \ldots$,}
\end{equation*}

in which $\Phi_{0} = I_{K}$ and $A_{j} = 0$ for $j > p$.

\vspace{3.5mm}
We noted that we desire the VMA form of the model because:
\begin{itemize}
 \item It is most convenient to calculate variances in this form because the VMA is a linear combination
       of indenpendent terms. Thus, there are no covariance terms in the calculations.
 \item We would like to interpret the $\phi_{ij}$ elements of the $\Phi_{i}$ matrices as dynamic multipliers 
       and conduct impulse response analysis. 
\end{itemize}


\vspace{10mm}
\textbf{Structural Vector Autoregressive Models (SVARs)} \\

\vspace{3.5mm}
We noted that before we could interpret the $\phi_{ij}$'s as dynamic multipliers that we typically need
to impose some structure on the VAR model. That is because unless the variance-covariance matrix, $\Omega$, 
is diagonal, the $\phi_{ij}$ cannot be cleanly interpreted as an exogenous unexpected shock to the $u_{it}$.


\vspace{3.5mm}
We specify an SVAR$(p)$ as follows:

\vspace{2mm}
\begin{equation*}
B y_{t} = \gamma + \Gamma_{1} y_{t-1} + \Gamma_{2} y_{t-2} + \cdots + \Gamma_{p} y_{t-p} + \epsilon_{t}
\end{equation*}

The structural variance-covariance matrix is $D$, which is diagonal. This gives us the following mapping
from reduced-form to structural parameters:

\vspace{2mm}
\begin{eqnarray*}
&& \nu = B^{-1} \gamma \\
&& A_{i} = B^{-1} \Gamma_{i} \mbox{, \quad for i = 1, 2, \ldots, p} \\
&& \Omega = B^{-1} D B^{-1 \prime}
\end{eqnarray*}

\vspace{3.5mm}
\textbf{Notes:} Typically, without some kind of restrictions the structural parameters are not identified
by the reduced-form parameters.

\vspace{3.5mm}
For $K=2$ (bivariate) an SVAR$(p)$ model in matrix form is the following:

\vspace{3mm}
\begin{equation*}
\begin{bmatrix} b_{11} & b_{12} \\ b_{21} & b_{22} \end{bmatrix} \times 
\left[ \begin{array}{c} y_{1t} \\ y_{2t} \end{array} \right] = \left[\begin{array}{c} \gamma_{1} \\ \gamma_{2} \end{array}\right]
+ \begin{bmatrix} \gamma_{11}^{(1)} & \gamma_{12}^{(1)} \\ \gamma_{21}^{(1)} & \gamma_{22}^{(1)} \end{bmatrix} \times 
\left[\begin{array}{c} y_{1t-1} \\ y_{2t-1} \end{array} \right] + 
\cdots + 
\begin{bmatrix} \gamma_{11}^{(p)} & \gamma_{12}^{(p)} \\ \gamma_{21}^{(p)} & \gamma_{22}^{(p)} \end{bmatrix} \times
\left[\begin{array}{c} y_{1t-p} \\ y_{2t-p} \end{array} \right] +  
\left[\begin{array}{c} \epsilon_{1t} \\ \epsilon_{2t} \end{array} \right]
\end{equation*}


\newpage
One way that we discussed to impose some structure on a VAR is to estimate a VAR in Sim's triangular form
(for an SVAR$(1)$ model):

\vspace{3mm}
\begin{align*}
y_{1t} &= \gamma_{1} + \gamma_{11}^{(1)} y_{1t-1} + \gamma_{12}^{(1)} y_{2t-1} + \epsilon_{1t} \\
y_{2t} &= \gamma_{2} - b_{21} y_{1t} + \gamma_{21}^{(1)} y_{1t-1} + \gamma_{22}^{(1)} y_{2t-1} + \epsilon_{2t} \\
\end{align*}

\vspace{3.5mm}
We noted that the algebra of OLS will guarantee that the variance-covariance matrix of the model in this form
will be diagonal.  The structural matrix $B$ from this model will look like the following:

\begin{equation*}
B = \begin{bmatrix} 1 & 0 \\ b_{21} & 1 \end{bmatrix}
\end{equation*}

\vspace{3.5mm}
\textbf{Note:} Sim's triangular form imposes a zero restriction on the parameter $b_{12}$. This is enough 
structure to allow us to identify the structural parameters.

\vspace{5mm}
We note that the triangular form imposes a recursive causal order.  That is, the variable $y_{1t}$ has a contemporaneous
effect on the variable $y_{2t}$, but $y_{2t}$ does not have a contemporaneous effect on $y_{1t}$. With the model in this
form we could proceed to estimatation equation-by-equation with OLS.

\vspace{10mm}
\textbf{Identification of the SVAR from the Reduced-Form VAR and the Cholesky Decomposition} \\

\vspace{3.5mm}
The triangular matrix decomposition (a special case of the Cholesky decomposition) decomposes a 
positive semi-definite matrix as follows:\

\begin{equation*}
\Omega = T \Lambda T^{\prime}
\end{equation*}

\vspace{3.5mm}
For which, 
\begin{equation*}
T = \begin{bmatrix} 1 & 0 \\ t_{21} & 1 \end{bmatrix} \mbox{, \quad} \Lambda = \begin{bmatrix} \lambda_{1} & 0 \\ 0 & \lambda_{2} \end{bmatrix}
\end{equation*}

\vspace{3.5mm}
We can use this result to identify the SVAR model as follows:

\begin{itemize}
 \item Estimate each equation of the reduced-form VAR with OLS.
 \item Apply the triangular decomposition to the estimated variance-covariance matrix.
\end{itemize}

\newpage
This gives us the following psuedo-structural VAR:

\begin{align*}
 T^{-1} y_{t} &= T^{-1} \nu + T^{-1} A_{1} y_{t-1} + \cdots + T^{-1} A_{p} y_{t-p} + T^{-1} u_{t} \\
 B y_{t} &= \gamma + \Gamma_{1} y_{t-1} + \cdots + \Gamma_{p} y_{t-p} + \epsilon_{t} \\
\end{align*}

Which one can see is equivalent to the SVAR model. The important thing to note here is that the order
of the variables as they enter the reduced-form VAR model, $y_{t} = (y_{1t},  y_{2t})^{\prime}$, determines
the recursive causal ordering of the identified SVAR.

\vspace{10mm}
\textbf{Structural Vector Moving Average Form} \\

\vspace{3.5mm}
Just as with the reduced-form VAR, the stationary SVAR model implies an infinite-order structural vector moving
average form:

\vspace{2mm}
\begin{align*}
y_{t} &= \sum\limits_{i=0}^{\infty} \Theta_{i} \gamma + \sum\limits_{i=0}^{\infty} \Theta_{i} \epsilon_{t-i} \\
      &= \mu + \sum\limits_{i=0}^{\infty} \Theta_{i} \epsilon_{t-i} \\
\end{align*}

\vspace{3.5mm}
The $\Theta_{j}$ matrices can be obtained from the following recursions:

\begin{equation*}
\Theta_{j} = \Phi_{j} B^{-1} \mbox{, \quad for $i = 1, 2, \ldots$,}
\end{equation*}

\vspace{3.5mm}
Now we can interpret the $\theta_{ij}$ parameters as dynamic multipliers and conduct policy analysis with
impulse response functions and forecast error variance decompositions.






























\end{document}
